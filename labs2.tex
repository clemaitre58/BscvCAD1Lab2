\documentclass[12pt]{TDTP}


\newcommand{\auteur}{C\'edric Lemaitre}
\newcommand{\couriel}{c.lemaitre58@gmail.com}
\newcommand{\promo}{Bachelor in Computer Vision}
\newcommand{\annee}{2018-2019}
\newcommand{\matiere}{Computer Aided Design 1}

\newcommand{\tdtp}{Practice}
\renewcommand{\sujet}{Intro to Matlab}


\begin{document}
\titre

\textbf{NOTE}\\
For each problem you shall create a script, for example \texttt{problem1.m}, containing all commands to answer the questions.


%%%%%%%%%%%%
\Exo
Find a MATLAB one-line expression to cretae the $n\times n$ matrix $A$ satisfying
$$
a_{ij} = 
\begin{cases}
   1 & \text{if } i-j \text{ is prime} \\
   0  & \text{otherwise}
 \end{cases}
$$

%%%%%%%%%%%%
\Exo
\textbf{Friday  the 13th}\\
Friday the 13th is unlucky (in many cultures), but is it unlikely?
What's the probability that the 13th day of any month falls on a Friday?
The quick answer is 1/7, but this is not quite right.

Write a MATLAB code that counts the number of times that Friday occurs on the various weekdays in a 400 year Gregorian calendar cycle, for example from the year 1601 to the yera 2000.

\bigskip
\textbf{NOTE}:\\
The MATLAB function \textbf{clock} returns a six-element vector $c$ with elements
\begin{verbatim}
c(1) = year
c(2) = month
c(3) = day
c(4) = hour
c(5) = minute
c(6) = seconds
\end{verbatim}

The first five elements are integers, while the sixth element has a fractional part that is accurate to milliseconds. 
The best way to print a \textbf{clock} vector is to use \textbf{fprintf} or \textbf{sprintf} with a specified format string that has both integer and floating point fields.

\begin{verbatim}
f = '%6d %6d %6d %6d %6d %9.3f\n'
\end{verbatim}

On September 10th, 2014 at 3:15 pm, the following commands

\begin{verbatim}
c = clock;
fprintf(f,c);
\end{verbatim}

produces

\begin{verbatim}
2014	9	10	15	15	30.543
\end{verbatim}

In otrher words,

\begin{verbatim}
year = 2014
month = 9
day = 10
hour = 15
minute = 15
seconds = 30.543
\end{verbatim}

The MATLAB functions \textbf{datenum}, \textbf{datevec}, \textbf{datestr}, and \textbf{weekday} use \textbf{clock} and facts about the Gregorian calendar to facilitate computations involving calendar dates.
You migh want to use them to solve the problem.

%%%%%%%%%%%%
\Exo
Given the following function
$$
s = a \cos(\phi) + \sqrt{b^2 - (a\sin(\phi)-c)^2}
$$
Plot $s$ as a function of the angle $\phi$ when $a=1$, $b=1.5$, $c=0.3$, and $0\leq \phi \leq 360°$.

%%%%%%%%%%%%
\Exo
Plot the following parametric functions (you will use the \textbf{axis equal }command after your \textbf{plot} command to force MATLAB to make the x-axis and y-axis the same lenght):
\begin{itemize}
\item A circle of radius 5
\item \textit{Leminscate} ($-\pi/4 \leq \phi \leq \pi/4$)
\begin{align*}
x &= \cos(\phi) \sqrt{2\cos(2\phi)} \\
y &= \sin(\phi) \sqrt{2\cos(2\phi)}
\end{align*}

\item \textit{Logarithmic Spiral} ($0 \leq \phi \leq 6\pi; \; k=0.1$)
\begin{align*}
x &= e^{k\phi}\cos(\phi) \\
y &= e^{k\phi}\sin(\phi) 
\end{align*}

\end{itemize}
%%%%%%%%%%%%
\Exo
Plot the following 3D curves using \textbf{plot3} function:
\begin{itemize}
\item \textit{Spherical helix}
\begin{align*}
x &= \sin(\frac{t}{2c}) \cos(t) \\
y &= \sin(\frac{t}{2c}) \sin(t) \\
z &= \cos(\frac{t}{2c})
\end{align*}
where $c=5$ and $0\leq t \leq 10\pi$.

\item \textit{Sine wave on a sphere}
\begin{align*}
x &= \cos(t) \sqrt{b^2 -c^2 \cos^2(at)} \\
y &= \sin(t) \sqrt{b^2 -c^2 \cos^2(at)} \\
z &= c*\cos(at)
\end{align*}
where $a=10$, $b=1$, $c=0.3$, and $0 \leq t \leq 2\pi$.
\end{itemize}

%%%%%%%%%%%%

%%%%%%%%%%%%%%%%%%%%%%%%%%%%%%%%%%%%%%%%
\end{document}
